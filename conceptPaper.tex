\documentclass[dvips,12pt]{article}

\usepackage[pdftex]{graphicx}


\setlength{\oddsidemargin}{0.25in}
\setlength{\textwidth}{6.5in}
\setlength{\topmargin}{0in}
\setlength{\textheight}{8.5in}

\title{Compilation of a CSP-like language}
\author{Ssendikadiwa Stephen}
\date{\today}

\begin{document}
	
	
	
	\maketitle
	
	\section{Introduction}
	
	The parallel programming language Occam is essentially an implementable sublanguage of CSP. Occam is a concurrent programming language that builds on the communicating sequential processes (CSP), process algebra \cite{ ("occam (programming language)," n.d.)}, and shares many of its features 
	
	\section{Background to the Problem }
	Most of the programming languages in the computing world have a very heavy syntax and wide grammar, lack parallelism and more so not portable. These programs run on a few clusters, servers and embedded systems.
	
	More so, they lack an extension to Occam which permits recursion and still lack virtual machines which are fully optimized for displaying a simulation, and translating the virtual machine code into native code for a real machine. 
	
	\section{Problem Statement}
	The problem this project will address is to produce a small portable implementation of a subset of Occam, to implement a virtual machine based on the inmos transputer, and a compiler which will target the language.
	The proposed system will implement an extension to Occam which permits recursion and separate virtual machines which are fully optimized for displaying a simulation.
	
	\section{Objectives}
	\subsection{Main Objective}
	The aim of this project is to produce a small portable implementation of a subset of Occam; the proposed technique is to implement a virtual machine based on the inmos transputer, and a compiler which targets that language
	
	\subsection{Other Objectives}
	
	\begin{itemize}
		\item To gather and analyze requirements that will be used in the design and development of a virtual machine based on the inmos transputer, and a compiler which targets that language.
		\item To implement the designed system
		\item To test and validate the designed system 
	\end{itemize}
	
	\section{Methodology}
	In order to  implement a virtual machine based on the inmos transputer, and a compiler which targets that language, we need to collect user requirements using various tools and techniques to achieve our objectives for example requirements analysis, interviews and literature review of existing systems.
	We will analyze our data collected using various methods such as process modeling specification which will include dataflow diagrams and context diagrams. We will also use process specification methods during analysis of requirements for example action diagrams. 
	\section{Outcomes}
	
	\section{References}
	
	
\end{document}